% write your paper in here

\chapter{Hi-C scaffolding of the bdelloid rotifer \textit{Adineta vaga}}

Bdelloid rotifers have been drawing interest due to their unusual ancient asexuality. A first diploid assembly of \textit{Adineta vaga} was published in 2013 \cite{flot2013}, with a total length of 218 Megabases (Mb) and a N50 of 260 kilobases (kb), and the genome was described as "incompatible with conventional meiosis". In the following paper, the genome of \textit{Adineta vaga} was assembled \textit{de novo} using PacBio CLR, Nanopore reads, Illumina reads and Hi-C reads. Three assemblies were produced: a collapsed haploid assembly using all types of reads; and two diploid assemblies, one using PacBio CLR obtained with FALCON-Unzip, and the second one using Illumina reads obtained with Bwise. All these assemblies were scaffolded with instaGRAAL, and converged to 6 haploid chromosomes (collapsed assembly) or 12 phased chromosomes (FALCON-Unzip and Bwise assemblies). These results show that the genome of \textit{Adineta vaga} is: diploid, with 6 pairs of chromosomes; a paleotetraploid, as it has homoeologous chromosomes (pairs 1, 2 and 3 are homoeologous to pairs 4, 5, 6 respectively); and compatible with meiosis. \\
I contributed to this study in the assembly and Hi-C scaffolding of the collapsed haploid assembly. 

%\begin{table}
%\caption{Comparison of assembly statistics with other genomes of the phylum Rotifera.}
%\resizebox{\columnwidth}{!}{
%\begin{tabular}{lllcccc}
%\hline
%\multirow{2}{*}{\textbf{Class}} & \multirow{2}{*}{\textbf{Species}} & \multirow{2}{*}{\textbf{Reads technology}} & %\textbf{Assembly} & \multirow{2}{*}{\textbf{N50}} & \multicolumn{2}{c}{\textbf{BUSCO}} \\
%    & & & \textbf{size} & & \textbf{single} & \textbf{dup.} \\
%\hline
%Bdelloida & \textit{Adineta vaga} & Illumina, Nanopore, PacBio, Hi-C & 101 Mb & 16.7 Mb &  &  \\
%    & \textit{Adineta vaga} \cite{flot2013} & Illumina, 454, mate pair & 218 Mb & 260 kb & 22.4\% & 64.8\% \\
%\hline
%Monogononta & \textit{Brachionus asplanchnoidis} \cite{brachionus_asplanchnoidis} & Illumina & 114 Mb & 10 kb & 82.O\% & 1.2\% \\
%    & \textit{Brachionus asplanchnoidis} \cite{brachionus_asplanchnoidis} & Illumina & 115 Mb & 9 kb &  &  \\
%\hline
%\end{tabular}
%}
%\label{tab:anthozoans}
%\end{table}