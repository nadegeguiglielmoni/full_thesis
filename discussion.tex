\chapter{Discussion \& Conclusion}

\section{Combining long reads and Hi-C for chromosome-level assemblies}

Long reads and Hi-C are undoubtedly a winning combination to obtain chromosome-level assemblies with high completeness. Long reads have disrupted the field of genome assembly and brought compelling new techniques in addition to short-read sequencing. Long-read sequencing is more laborious than short reads due to its requirement for high-molecular-weight DNA, yet long-read assemblies are generally favored for their higher contiguity and better resolution of repeats. However, long reads are often not sufficient to assemble eukaryote genomes into chromosome-level contigs. Increasing the sequencing depth may improve the contiguity to a certain extent, but long-read assemblers do not seem able to take advantage of huge sequencing depths (up to 230X of PacBio CLR and 170X of Nanopore reads in the case of \textit{Adineta vaga}) to fully solve assemblies. Scaffolding is still a necessary step, and Hi-C has emerged as the most robust method to bring assemblies to chromosome level. The popularity of Hi-C has stimulated the release of protocols, commercial kits and tools, providing researchers with a variety of options to adapt to their genome projects. The genomes presented here, \textit{Adineta vaga}, \textit{Astrangia poculata} and \textit{Flaccisagitta enflata}, were assembled with a mix of short reads, long reads and Hi-C, and all reached chromosome-level scaffolds with high completeness. The genome assembly of \textit{Flaccisagitta enflata} was the most challenging out of the three due to: its moderate size (694-699 Mb); its high heterozygosity; the low N50 of Nanopore reads; the poor Hi-C mapping rate. Nevertheless, the quality of the final assembly further demonstrates the robustness of the combination of long reads and Hi-C. \\

The amount of Hi-C data and the mapping rates are highly variable among projects (Table \ref{tab:hic_data}). The differences in mapping rates cannot be attributed to read length as the Hi-C reads for \textit{Adineta vaga} are only 66 bases-long (against 150 bases for \textit{Astrangia poculata} and \textit{Flaccisagitta enflata}), but \textit{Adineta vaga} has the highest mapping rate (83\%). In addition, most Hi-C reads of \textit{Adineta vaga} (72\%) were mapped in the first round of iterative mapping, using only 20 bases. The low mapping rate for \textit{Flaccisagitta enflata} may be attributed to the high heterozygosity of the genome and to the use of a different individual than the one used for Illumina and Nanopore sequencing. It is unclear what would be the necessary amount of reads for Hi-C scaffolding to obtain chromosome-level scaffolds. The company Arima Genomics recommends 200 millions pairs of Hi-C reads for a 1-Gb genome. This estimation does not take into account the mapping rate nor the fragmentation of the genome, and a thorough review of Hi-C scaffolding should consider these factors to find optimal Hi-C sequencing depths depending on the genome projects. \\

Furthermore, Hi-C reads are generated for scaffolding in genome projects, but they can also serve to study genome architecture. As chromosome-level assemblies and Hi-C datasets are accumulating for a wide variety of species, these resources could be compiled into an evolutionary analysis based on the 3D genomes. For instance, a recent study investigated the mechanisms underlying genome folding in 27 species of animals, fungi and plants \cite{hic_genomes}. This analysis targeted eukaryotes in general; it surveyed 20 animals, including 6 vertebrates, and disregarded several metazoan phyla. Recently published non-vertebrate genomes with Hi-C data, such as the sponge \textit{Ephydatia muelleri} \cite{ephydatia_mulleri}, and the ones presented here, could be integrated in a large study of the 3D genomes of animals. \\

\begin{table}
\centering
\begin{tabular}{lcc}
\hline
\textbf{Species} & \textbf{\# Hi-C pairs} & \textbf{Mapping rate} \\
\hline
\textit{Adineta vaga} & 55 millions & 83\% \\
\textit{Astrangia poculata} & 723 millions & 67\% \\
\textit{Flaccisagitta enflata} & 489 millions & 37\% \\
\hline
\end{tabular}
\caption{Overview of Hi-C datasets.}
\label{tab:hic_data}
\end{table}

\section{Defining a new benchmark dataset: \textit{Adineta vaga}}

New assembly tools are generally tested on bacteria and model organisms with a low heterozygosity, such as \textit{Drosophila melanogaster}, \textit{Caenorhabditis elegans}, \textit{Homo sapiens}, and up to a heterozygosity of 1\% for \textit{Arabidopsis thaliana}. Testing programs on a human is often a requirement for publication, as large sequencing datasets of all types are available and this genome is the closest to a perfect assembly, since the release of a gap-less reference \cite{complete_human}. Subsequently, tools are often tuned for low-heterozygosity genomes and can only poorly handle higher levels of heterozygosity. The benchmark of long-read assemblers (Chapter 2) shed light on the limitations of these assemblers on a non-model genome, \textit{Adineta vaga}, with a mixture of low-heterozygosity and high-heterozygosity regions. These assemblers can still lead to high-quality assemblies when combined with pre-assembly filtering or post-assembly haplotig purging. The genome of \textit{Adineta vaga} now has a reliable reference and large PacBio CLR, Nanopore, Illumina, and Hi-C datasets, making it a compelling example for benchmarks of new assembly tools on a mid-heterozygosity genome. These assembly strategies are not exclusive to \textit{Adineta vaga}, and some were used for \textit{Astrangia poculata} and \textit{Flaccisagitta enflata}. Implementing this methodology into an assembly pipeline, including evaluation steps to identify well-collapsed candidate assemblies, would facilitate assembly projects. \\

\section{Decreasing long-read sequencing depth}

wtdbg2 emerged as the most cost-effective assembler, reaching a size close to the expected genome size with as little as a 10X long-read dataset, and with small variations with increasing sequencing depth. This result on \textit{Adineta vaga} was further confirmed with the genome of \textit{Astrangia poculata}: wtdbg2 yielded the best draft assembly using a 15X Nanopore dataset. Besides, wtdbg2 requires small computational resources compared to other assemblers. The capacity of wtdbg2 to accommodate low-depth long-read datasets demonstrates that the limiting factor is not sequencing depth but long-read assemblers. The cost of sequencing and assembly is crucial as it will determine the feasibility of a genome project, thus adapting assemblers to low sequencing depth (around 10 to 20X) would decrease the financial burden of genome assembly and increase accessibility for any research team. \\

\section{Phasing assemblies}

As chromosome-level scaffolds have become common, the challenge of genome assembly is now moving on to another step: phasing assemblies. This goal brings new difficulties: phasing is incompatible with pooling individuals (unless these are clones), and the necessary sequencing depth is multiplied by the ploidy of the genome at hand. GraphUnzip offers an approach for phasing genomes with long reads and Hi-C. It requires an uncollapsed assembly rather than trying to call variants from a collapsed assembly, thus it is adequate for non-model genomes, with large heterozygous genomes and sometimes hemiploidy. However, GraphUnzip need additional tests to evaluate the correct association of haplotypes from one heterozygous region to another. PacBio HiFi reads are opening new possibilities for phased assemblies thanks to their low error rate, hence we can expect full haplotype-resolved assemblies to become common in the next years. \\

\section{Reproducibility in genome projects}

As more and more genomes are being released, protocols are published in parallel which circumvent the challenges posed by a given species, and these methods can be applied for genome projects of similar species. Many genome projects have recourse to private companies, providing a service for high-molecular-weight DNA extraction, long-read sequencing, and Hi-C sequencing. As a result, the protocols are not available publicly; while having these chromosome-level assemblies is essential, they do not bring clues for new genome projects. This is the case for the genome of \textit{Astrangia poculata}, as we performed Nanopore sequencing, but high-molecular-weight DNA extraction and Hi-C sequencing were done by Dovetails Genomics, and other stony coral projects cannot build on our work. For \textit{Adineta vaga} and \textit{Flaccisagitta enflata} however, the methods are publicly available and can be reproduced. In addition, sequencing platforms use in-house tools; some are open source (e.g. the PacBio CLR assembler Falcon, the PacBio HiFi assembler IPA for Pacific Biosciences), whereas others are not (the Hi-C assembler HiRise, for Dovetails Genomics, only has a non-user-friendly and outdated version online). These programs may often result in high-quality assemblies, yet, depending on the genome, some open-source alternatives may be more adequate. \\

Genome assemblies presented in Chapter 1 were surveyed on the National Center for Biotechnology Information (NBCI) database \cite{ncbi}, which provides information on genome size, contig N50 and scaffold N50. Not all assemblies found in publications were available on this website, and in some cases, the assembly statistics in the paper did not match the ones on the database. For most projects, sequencing datasets were associated with a BioProject number, but these datasets were sometimes incomplete or missing. The absence of these sequencing datasets prevents reusability, reproducibility, or independant analyses. The metadata also lack information on the tools used for assembly; comprehensive records of assembly methods would help evaluate assembly programs. \\
